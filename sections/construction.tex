The Tarot 680 Pro hexacopter is the base platform selected for this project. The large (680mm radius), hexacopter design allows it to maintain its stability in relatively high winds. Its Tarot 4108 motors spin 350mm propellers which allow the drone to carry extra instruments for applications after this project. The foldable, carbon fiber frame provides structural robustness and portability. It has two batteries in order to create two isolated power systems - one for the computational and navigational electronics, and one for the propulsion system and gimbal. The power isolation prevents the motors and gimbal from causing electrical spikes which can damage or disable the computational electronics. The gimbal allows the drone to capture a stable video image over a wide angular range. 

Joshua and Garret carried out the initial construction process slowly and carefully, placing emphasis on reliability in all physical and electrical connections. This is to avoid expensive crashes by minimizing the risk of in-flight component failure. First, we mounted a motor to each of the motor arms. We soldered bullet connectors to the motor leads to attach them to the speed controllers. We mounted speed controller to the bottom of each motor mount using a 3D-printed bracket. We soldered bullet connectors to the speed controller wires, and wire extensions (also with bullet connectors) were run through the hollow arm tubes because the speed controller wires alone were too short. After all motors were verified to spin the expected direction according to the ArduPilot documentation, we placed heat shrink around all bullet connector joints for electrical insulation and physical strengthening of the connection. We assembled the drone bodies and placed the arms into their mounting points. We then fastened landing gear to the drone body. After this, we installed the computational and navigational electronics were to the mounting plate. Joshua installed the operating systems for the flight controller and companion boards, along with all necessary software (to be discussed later). ArduPilot was set up to run as a system service on the flight controller, and we carried out all necessary calibration of the accelerometers, gyroscopes, magnetometer, and RC radio according to the ArduPilot guidelines \cite{ardupilot_setup}.

% The base platform we decided to use was the Tarot 680 Pro. Construction began with individual arms, mounting the motors and their base plate to the carbon tube arms, with wires protruding from the underside. We then assembled the landing gear which was mounted with set screws into the t-joint and rubber caps place on the ends. The landing gear mount was then assembled with the tube inside, sandwiched between two mounting plates and the mount that would attach the bottom center plate of the drone. We then added the supports for the gimbal and battery mount in the form of small metal rings, which were screwed into the bottom plate, and placed rubber grommets inside them to hold the tube rails and provide them some minor dampening. plastic clips were then attached to the motor arms and mounted to the top of the bottom plate. The horizontal motor arms were not mounted with clips and would come slightly later. More clips were screwed into the top of the bottom plate such that the arms can snap into them, allowing them to fold. At this point we took all of the motor arms back off, having confirmed how they fit, and moved on to soldering them to the top board to receive power. Before we could do that, however, we needed to attach the speed controllers to the motors and arms. We first placed some heat-shrink over the wires leading out of the motors, before soldering male bullet connectors to the wire which plugs in to the speed controller. The wires from the speed controllers were not long enough to reach all the way through the carbon tube so we sourced some extra wire and used more bullet connectors and heat shrink to extend them along with a servo extension. We then trimmed the extension on the other side of the carbon tube to leave a little bit of play so that it could flex during flight and the folding of the arms before soldering them to the power board. After which, we also soldered the XT90 power connector to the board. The next process was to attach the top and bottom plates together and sandwich the final two motor arms into their metal clamps. Once that was all together, the frame was assembled and we could move on to mounting the electronics. We 3d printed another plate which sits on top of the power board in order to provide convenient mounting points for our electronics. We then mounted and connected the motors to the flight controller on their corresponding channels and then telemetry through SBUS. After testing the motors to make sure everything was configured correctly and spinning in the right direction, we then finalized our connections with the heat-shrink. We then confirmed our mounting points for the battery attached to the undercarriage of the drone with Velcro straps. The battery we are using is a 6s, 10,000mah. Now that everything has power, is configured correctly, and securely mounted, we are almost ready for the maiden flight. The only thing left to configure is telemetry and its connection to our radio transmitter. Once bound we tested the drone for the first time. *****Maybe point to the results section here***** After testing the mechanical and manual flight aspects of our drone, we could move on to configuring and mounting the gimbal and jetson-nano. The gimbal being used is a T4-3D tarot gimbal designed for the GoPro Hero 3. We designed and 3d printed a camera mount to work within these dimensions. After working with several drivers and firmware updates, everything was configured on the gimbal so we could begin manual testing. To do this we bound the gimbal to our radio transmitter. We discovered several inconsistencies with the gimbal struggling to maintain extreme angles. In response, we limited the maximum angle that it could acheive. From there the gimbal is connected to the Navio which is connected to the Jetson Nano which we now mounted to the top plate. The Navio is connected to the Jetson Nano via MAVROS over Ethernet. The Jetson Nano would then process the image from the gimbal camera and tell the Navio to provide PWM signals in order to control the gimbal to look at the April Tag or WhyCon marker. 