This project has demonstrated that fiducial markers can be used in combination with a gimbal-mounted camera to localize and approach a landing pad reliably, even without knowing the orientation of the gimbal. It is a proof of concept of many of the points addressed in simulation in the original thesis \cite{AL_thesis}. While more testing is needed to determine a better landing policy, and more development is needed for the WhyCode marker, nonetheless we have succeeded in building a drone that is capable of both manual and autonomous flight, real-time video analysis, and fiducial marker identification and tracking.

We have started communication with the maintainers of WhyCode in the Czech Republic and will continue to integrate WhyCode markers into our system. We will also adjust the landing control policy to reflect new knowledge of the performance of the drones during landing, and this will allow us to carry out successful landings in the near future. We will also focus on solving the power issues experienced with the Jetson Nano. Eventually we will test new methods of pose estimation and control, with a focus on artificial intelligence and machine learning.