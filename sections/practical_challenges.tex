\subsection{Supply Chain Issues}

The Tarot 680 kit uses 28mm M2.5 machine screws which are atypical for Iceland and could not be found. Moreover, half of such necessary screws were missing from the drone kits (4 missing for each drone). This meant that only one drone could initially be assembled until a colleague graciously retrieved some acceptable 30mm M2.5 machine screws from mainland Europe.

The Google Coral camera module uses a smaller ribbon cable than those compatible with the Raspberry Pi and similar boards. The short ribbon cable that comes with the camera module is unusable in this application because it does not allow the gimbal to move freely throughout its entire range of motion. To solve this, several terminals were delicately removed by hand from an existing, longer cable with the same terminal size. This cable was then successfully used both in the lab and the field to provide good connectivity between the Google Coral and its camera module.

In general, the main connector used between components is a USB cable. However, the small electronics compartment and the large number of components sitting therein meant that a lot of these cables had to be modified to fit with the canopy completely closed. The stiff, plastic ends of some of these cables were cut in order to make them take up less room, particularly in the case of the Jetson Nano whose micro USB port was quite close to the canopy walls. Additionally, many of these cables were shortened by cutting and re-soldering the wires to reduce space and weight.

\subsection{ROS Installation}

Installing the ROS software stack on the Jetson Nano and Google Coral can be a challenge because of the number of dependencies and the specific operating systems which do not have as much popularity as typical Linux distributions that ROS targets, such as basic Ubuntu. Significant time was spent creating reliable installation instructions for both boards. Such instructions were a necessary step, as a single working OS image is not adequate for the project since SD cards or installations may otherwise fail, and the system may need to be reinstalled. On-board compilation of ROS sources takes a long time but is ultimately not prohibitively time-consuming.

The Jetson Nano runs a modified version of Ubuntu 18.04 called Tegra, which comes with a lot of desktop applications and games that were removed prior to the installation of ROS. These take up a large portion of the system image that is provided by NVIDIA, and they are completely unnecessary. Many of the necessary ROS modules use OpenCV, but are incompatible with OpenCV 4 which is provided on this system image. Therefore, OpenCV was downgraded to version 3.2 for compatibility. Many of the ROS dependencies can be installed through the Aptitude package manager on this board, but the \texttt{whycon\_ros}, \texttt{jetson\_csi\_cam}, \texttt{gscam}, \texttt{gimbal\_controller}, and \texttt{landing\_controller} modules were necessarily installed from source. After many attempts at installation, with multiple ROS distributions, this method of combining binary and source installations worked on the board with the Melodic ROS distribution.

The Google Coral runs a modified version of Ubuntu 18.04 called Mendel Linux, which is much more lightweight than Tegra. The Coral's 8 GB of fast, onboard storage means that the Coral runs generally more smoothly and boots more quickly than the Jetson Nano, which boots from an SD card. However, an SD card was eventually needed for the ROS installation. This installation was done entirely from source, although some binaries are provided through Aptitude. A swap file is needed both for ROS installation and initialization.

The ROS Melodic distribution was successfully installed on both boards, using dependencies for Ubuntu 18.04 by setting the \texttt{ROS\_OS\_OVERRIDE} environment variable to \texttt{ubuntu:18.04:bionic}.

\subsection{Jetson Nano Power Consumption and Form Factor}
\label{section:jetson_nano_power_consumption_and_form_factor}

Although the Jetson Nano itself can run without a problem given the setup of the electronic system onboard its drone, the camera adds too much to the power requirements. The result is that the Jetson Nano is unable to sustain operation reliably. Although in lab scenarios it is generally able to successfully identify the landing pad, aim the gimbal, and generate a positional target based on the landing pad's pose, it tends to shut off when the computational load becomes high. Moreover, the shutoff is akin to a brown-out in that the system voltage gradually decreases until the board no longer functions. This led to the failure of multiple SD cards which the board uses as its main storage and boot disk. A lot of overhead hours were spent identifying this issue and determining eventually that the Jetson Nano is not suited for this application because of its power requirements.

The relatively large form factor of the Jetson Nano also means that it takes up a lot of valuable space in the electronics compartment - both horizontally on the mounting plate, and vertically towards the top of the canopy (as shown in Figure \ref{fig:jetson_electronics}). This is especially so when using the recommended fan which mounts on top of the heat sink, which already takes up significant space.

\subsection{Gimbal Quirks}

The Tarot T-3D gimbal, used on both drones, is made for a Go Pro and calibrated for its weight. The 3D-printed camera cases and their mounted modules ware significantly lighter than a GoPro, and therefore created a weight imbalance in the roll axis of the gimbal, resulting in a higher-than-normal idle load on the gimbal roll motor. Conceptually this is not a problem, as the gimbal is still able to hold the camera smooth when the gimbal's IMU senses vibration or other movement. However, in static testing, the gimbal detects this higher effort and temporarily shuts off in order to attempt to save the gimbal. Normally the higher effort would correspond to some physical block to the gimbal's motion, and there is a risk of burning out the motor if current is continuously applied. To work around this behavioral quirk, and the closed source, black box gimbal firmware, additional weight was added to the camera case for balance.

Although the gimbal does calculate its orientation and can provide it to its firmware GUI on a Windows machine, it does not provide its orientation as data through an interface that can be read by the flight controller. Although it is theoretically possible to reverse-engineer the communication between the gimbal firmware and its IMU, this task is beyond the scope of the project and has the potential to be extremely time-consuming. A first attempt to estimate the gimbal's pan and tilt angles was moderately successful, as explained in Section \ref{section:gimbal_controller_adapation}. However, this attempt was ultimately abandoned because, although a simple PWM signal does communicate a target angle to the gimbal, the gimbal only approaches this angle over a non-negligible amount of time using its own PID controllers. This means that, if the gimbal controller calculates an angle based on the PWM signal that is sending to the gimbal, that calculated angle will have some non-negligible, transient error if the drone or the gimbal is in motion. Since both the drone and the gimbal are constantly in motion in the flight scenarios of this project, determination of the gimbal's orientation is avoided, and all pose estimation is done strictly from the pose of the WhyCon marker on the landing pad. Additionally, since the marker has no reliable yaw orientation, the pan (yaw) axis of the gimbal is fixed straight ahead via a static PWM signal.

\subsection{Field Testing Restrictions and Environmental Limitations}

A conservative testing methodology addresses the high risk of drone flight generally, as well as material scarcity and restrictions on drone flight at the campus of Reykjavik University. Testing was conducted outside the city of Reykjavik, and away from the airport, meaning that all flights were extremely focused with a clear objective, in order to reduce the risk of expensive and time-consuming crashes, battery consumption and travel overheads. In-lab testing was maximized. The prevalence of high winds and rainy days significantly limited the available testing time. Even on calmer days, overcast skies significantly decreased GPS accuracy and therefore manual control was very important.