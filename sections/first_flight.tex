% For our first flight, after finding a suitable location, we flew with manual control in order to test the PID controls on ArduCopter. We hovered the drone close to the ground and ran through all of the potential maneuvers it would need to complete. The stock PID configurations were very much adequate in all pitch, roll, yaw, and throttle inputs, responding quite well. After landing we went over the drone again to check the condition of everything, and confirmed that everything was still fine. After the test, which was conducted over two minutes, we found that it had used 700mah out of 10,000.

% The maiden flights of each drone were quite successful. The stock PID settings in the ArduPilot software were adequate for stable flight, even in winds of 5-10 m/s. The 
The first flights of each drone were conducted over about 2 minutes each. Both drones performed satisfactorily with basic hovering, horizontal and vertical movement, and yawing. The stock ArduPilot PID controller configurations on every axis provided reliable control with no oscillation. The drones were able to hold their positions in space even in the presence of wind. The frames were robust in both flight and landing. The energy used for each of the 2-minute flights was about 700 mAh from a total capacity of 10,000 mAh on the larger battery. This energy consumption rate was similar to all subsequent flights.